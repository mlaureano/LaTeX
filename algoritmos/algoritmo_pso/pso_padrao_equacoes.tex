% Copyright 2024 by Marcos Laureano (marcos.laureano@ifpr.edu.br)
% This file may be distributed and/or modified
%
% 1. under the LaTeX Project Public License and/or
% 2. under the GNU Public License.

\documentclass{book}
\usepackage[utf8]{inputenc}
\usepackage{amsmath,mathtools,lmodern,amssymb}


\usepackage[portuguese,algochapter,linesnumbered,inoutnumbered]{algorithm2e}

\SetKw{Kwpasso}{passo}

\SetKwProg{Fn}{Função}{}{}

\begin{document}
%    
%    
\begin{algorithm}
Passo 1. Inicialização\\
\Inicio{ 
   \ParaCada{partícula \(i \in 1,...,N_p\)}{
      Inicializa posição da partícula distribuído de forma uniforme sendo $P_i(0) \thicksim    U(LB,UB)$, onde LB e UB representa os limites da parte 
baixa e alta do espaço de busca\\
      \BlankLine
      Inicializa \emph{pbest} para a posição inicial: $pbest(i,0)=P_i(0)$\\
      Inicializa \emph{gbest} para o menor valor do enxame: $gbest(0)=arg min f[P_i(0)]$\\
      Inicializa a velocidade: $V_i\thicksim U(-|UB-LB|, |UB-LB|)$
   } %ParaCada
} %Inicio
\BlankLine
Passo 2. Repete até o critério ser alcançado\\
\Inicio{
   \ParaCada{partícula \(i \in 1,...,N_p\)}{
      Escolha aleatoriamente: $r_1,r_2 \thicksim U(0,1)$\\
      \BlankLine
      Atualiza a velocidade da partícula: $V_i(t+1) = \omega V_i(t) + c_1 r_1 ( pbest(i,t) - P_i(t) ) + c_2 r_2( gbest(t)-P_i(t) )$\\
      \BlankLine
      Atualiza a posição da partícula: $P_i(t+1)=P_i(t) +V_i(t+1)$\\
      \BlankLine
      \Se{$f[P_i(t)] < f[pbest(i,t)]$}{
         Atualiza a melhor posição conhecida da partícula i: $pbest(i,t) = P_i(t)$\\
         \Se{ $f[P_i(t)] < f[gbest(t)]$}{
            Atualiza a melhor posição do enxame: $gbest(t) = P_i(t)$
         }
         $t \leftarrow (t+i)$
      }
   }
}
\BlankLine
Passo 3. \Retorna $(gbest(t))$ como a melhor solução
\end{algorithm}
\end{document}