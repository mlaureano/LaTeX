\documentclass{article}
\usepackage[utf8]{inputenc}
\usepackage{booktabs,
	makecell,
	multirow,
	tabularx}

\usepackage{adjustbox}
\usepackage{fontawesome}


\newcolumntype{L}{>{\raggedright\arraybackslash}X}
\renewcommand\theadfont{\bfseries}
\renewcommand\theadgape{}
\renewcommand\cellrotangle{-90}
\usepackage{enumitem}

\begin{document}
\begin{table}
	\setlist[itemize]{nosep,wide}
	\setlist[1]{align=left,leftmargin=0.3cm,labelsep=0.5em,label=\normalfont\bfseries --,labelwidth=0.1cm}
	
	\centering
    \begingroup
    \fontsize{7pt}{7pt}\selectfont
	\begin{tabularx}{\linewidth}{@{}c L c L @{}}
        
		\toprule
		\makecell[bc]{Princípios\\ Gerais}
		& \makecell[bc]{Tentar Criar\\ superioridade numérica}
		& \makecell[bc]{Evitar a\\ igualdade\\ numérica}
		& \makecell[bc]{Não permitir a\\ inferioridade numérica}    \\
		\midrule
		Fases
		& \thead{Ataque (com posse de bola)}
		& \multirow{66}{*}{\rotatebox{-90}{\huge Transição Defesa-ataque e/ou Ataque-defesa}}
		& \thead{Defesa (sem posse de bola)}                        \\
		\cmidrule(r){1-2} \cmidrule(l){4-4}
		\makecell[l]{Princípios \\ Opera-\\ cionais}
		& \makecell[l]{Conservar a bola;\\
			Construir ações ofensivas;\\
			Progredir pelo campo do jogo adversário;\\
			Criar situações de finalização;\\
			Finalizar a baliza adversária.}
		&   & \makecell[l]{Impedir a progressão do adversário;\\
			Reduzir o espaço de jogo adversário;\\
			Proteger a baliza;\\
			Anular as situações de finalização;\\
			Recuperar a bola.}                           \\
		\cmidrule(r){1-2} \cmidrule(l){4-4}
		\rotatebox[origin=r]{-90}{\huge Princípios Fundamentais}
		& \begin{tabular}[b]{@{}>{\centering\arraybackslash}p{\linewidth}@{}}
			\thead{Penetração}
			\begin{itemize}
				\item Desestabilizar a organização adversária;
				\item Atacar diretamente o adversário ou baliza;
				\item Criar situações vantajosas para o ataque em termos numéricos e espaciais.
			\end{itemize}   \\
			\midrule
			\thead{Cobertura Ofensiva}
			\begin{itemize} 
				\item Dar apoio ao portador da bola oferecendo-lhe opções para a sequência do jogo;
				\item Diminuir a pressão adversária sobre o portador da bola;
				\item Criar superioridade numérica;
				\item Criar desequílibrio na organização defensiva adversária;
				\item Garantir a manutenção da posse de bola.
			\end{itemize}   \\
			\midrule
			\thead{Mobilidade}
			\begin{itemize}
				\item Criar ações de ruptura da organização defensiva adversária;
				\item Apresentar-se em um espaço muito propício para a consecução do gol;
				\item Criar linhas de passe em profundidade;
				\item Conseguir o domínio da bola para dar sequência a ação ofensiva (passe ou finalização).
			\end{itemize}   \\
			\midrule
			\thead{Espaço}
			\begin{itemize}
				\item Utilizar e ampliar o espaço de jogo efetivo da equipe;
				\item Expandir as distâncias/posicionamentos entre os jogadores adversários;
				\item Dificultar as ações de marcação da equipe adversária;
				\item Facilitar as ações ofensivas da equipe;
				\item Movimentar para um espaço de menor pressão;
				\item Ganhar "tempo" para tomar a decisão correta para dar sequência no jogo;
				\item Procurar opções mais seguras, através dos jogadores posicionados mais defensivamente, para dar sequência ao jogo.
			\end{itemize}   \\
			\midrule
			\thead{Unidade Ofensiva}
			\begin{itemize}
				\item Facilitar o deslocamento da equipe para o campo de jogo adversário;
				\item Permitir a equipe atacar em unidade ou em bloco;
				\item Oferecer mais segurança as ações ofensivas realizadas no centro do jogo;
				\item Propiciar que mais jogadores se posionem no centro do jogo;
				\item Diminuir o espaço de jogo no campo defensivo.
			\end{itemize}
		\end{tabular}
		&   &   \begin{tabular}[b]{@{}>{\centering\arraybackslash}p{\linewidth}@{}}
			\thead{Contenção}                                         
			\begin{itemize}
				\item Diminuir espaço de ação ofensiva do portador da bola;
				\item Orientar a progressão do portador da bola;
				\item Parar ou atrasar o ataque ou contra-ataque adversário;
				\item Propiciar maior tempo para organização defensiva;
				\item Restringir as possibilidades de passe a outro jogador adversário;
				\item Evitar o drible que favoreça progressão pelo campo de jogo em direção ao gol;
				\item Impedir a finalização à baliza.
			\end{itemize}    \\
			\midrule
			\thead{Cobertura defensiva}
			\begin{itemize}
				\item Servir de novo obstáculo ao portador da bola, caso esse passe pelo jogador de contenção;
				\item Transmitir segurança e confiança ao jogador de contenção para que ele tenha iniciativa de combate às ações ofensivas do portador da bola;
			\end{itemize}    \\
			\midrule
			\thead{Equilíbrio}
			\begin{itemize}
				\item Assegurar a estabilidade defensiva da região de disputa da bola;
				\item Apoiar os companheiros que executam as ações de contenção e cobertura defensiva;
				\item Cobrir eventuais linhas de passe;
				\item Marcar potenciais jogadores que podem receber a bola;
				\item Fazer recuperação defensiva sobre o portador da bola;
				\item Recuperar ou afastar a bola da zona onde ela se encontra.
			\end{itemize}    \\
			\midrule
			\thead{Concentração}
			\begin{itemize}
				\item Aumentar a proteção ao gol;
				\item Condicionar o jogo ofensivo adversário para zonas de menor risco do campo de jogo;
				\item Propiciar aumento de pressão no centro do jogo.
			\end{itemize}    \\
			\midrule
			\thead{Unidade defensiva}
			\begin{itemize}
				\item Permitir a equipe defender em unidade ou em bloco;
				\item Garantir estabilidade espacial e sincronia dinâmica entre as linhas longitudinais e transversais da equipe em ações ofensivas;
				\item Diminuir a amplitude ofensiva da equipe adversária na sua largura e profundidade;
				\item Assegurar linhas orientadoras básicas que influenciam as atitudes e os comportamentos tático-técnicos dos jogadores que se posicionam fora do centro de jogo;
				\item Equilibrar ou reequilibrar constantemente a repartição de forças da organização defensiva consoante às situações momentâneas de jogo;
				\item Reduzir o espaço de jogo utilizando a regra do impedimento;
				\item Obstruir  possíveis linhas de passe para jogadores que se encontram  fora do centro de jogo;
				\item Possibilitar a participação em uma ação defensiva subsequente;
				\item Propiciar que mais jogadores se posicionem no centro de jogo.
			\end{itemize}
		\end{tabular}    \\
		\bottomrule
	\end{tabularx}
    \endgroup
\end{table}

\pagebreak
    \begin{table}[p]
\setlist[itemize]{nosep,wide}
    \centering
\caption{Princípios Táticos do Jogo de Futebol}
    \tiny
\begin{tabularx}{\linewidth}{@{}c L c L @{}}
    \toprule
\makecell[bc]{Princípios\\ Gerais}
    & \makecell[bc]{Tentar Criar\\ superioridade numérica}
        & \makecell[bc]{Evitar a\\ igualdade\\ numérica}
            & \makecell[bc]{Não permitir a\\ inferioridade numérica}    \\
    \midrule
Fases
    & \thead{Ataque (com posse de bola)}
        & \multirow{66}{*}{\rotatebox{-90}{\huge Transição Defesa-ataque e/ou Ataque-defesa}}
            & \thead{Defesa (sem posse de bola)}                        \\
    \cmidrule(r){1-2} \cmidrule(l){4-4}
\makecell[l]{Princípios \\ Opera-\\ cionais}
    & \makecell[l]{Conservar a bola;\\
                   Construir ações ofensivas;\\
                   Progredir pelo campo do jogo adversário;\\
                   Criar situações de finalização;\\
                   Finalizar a baliza adversária.}
        &   & \makecell[l]{Impedir a progressão do adversário;\\
                           Reduzir o espaço de jogo adversário;\\
                           Proteger a baliza;\\
                           Anular as situações de finalização;\\
                           Recuperar a bola.}                           \\
    \cmidrule(r){1-2} \cmidrule(l){4-4}
\rotatebox[origin=r]{-90}{\huge Princípios Fundamentais}
    & \begin{tabular}[b]{@{}>{\centering\arraybackslash}p{\linewidth}@{}}
        \thead{Penetração}
    \begin{itemize}
    \item Desestabilizar a organização adversária;
    \item Atacar diretamente o adversário ou baliza;
    \item Criar situações vantajosas para o ataque em termos numéricos e espaciais.
    \end{itemize}   \\
    \midrule
        \thead{Cobertura Ofensiva}
    \begin{itemize}
    \item Dar apoio ao portador da bola oferecendo-lhe opções para a sequência do jogo;
    \item Diminuir a pressão adversária sobre o portador da bola;
    \item Criar superioridade numérica;
    \item Criar desequílibrio na organização defensiva adversária;
    \item Garantir a manutenção da posse de bola.
    \end{itemize}   \\
    \midrule
        \thead{Mobilidade}
    \begin{itemize}
    \item Criar ações de ruptura da organização defensiva adversária;
    \item Apresentar-se em um espaço muito propício para a consecução do gol;
    \item Criar linhas de passe em profundidade;
    \item Conseguir o domínio da bola para dar sequência a ação ofensiva (passe ou finalização).
    \end{itemize}   \\
    \midrule
        \thead{Espaço}
    \begin{itemize}
    \item Utilizar e ampliar o espaço de jogo efetivo da equipe;
    \item Expandir as distâncias/posicionamentos entre os jogadores adversários;
    \item Dificultar as ações de marcação da equipe adversária;
    \item Facilitar as ações ofensivas da equipe;
    \item Movimentar para um espaço de menor pressão;
    \item Ganhar "tempo" para tomar a decisão correta para dar sequência no jogo;
    \item Procurar opções mais seguras, através dos jogadores posicionados mais defensivamente, para dar sequência ao jogo.
        \end{itemize}   \\
    \midrule
        \thead{Unidade Ofensiva}
    \begin{itemize}
    \item Facilitar o deslocamento da equipe para o campo de jogo adversário;
    \item Permitir a equipe atacar em unidade ou em bloco;
    \item Oferecer mais segurança as ações ofensivas realizadas no centro do jogo;
    \item Propiciar que mais jogadores se posionem no centro do jogo;
    \item Diminuir o espaço de jogo no campo defensivo.
    \end{itemize}
    \end{tabular}
        &   &   \begin{tabular}[b]{@{}>{\centering\arraybackslash}p{\linewidth}@{}}
                \thead{Contenção}                                         \\
           \begin{itemize}
           \item Diminuir espaço de ação ofensiva do portador da bola;
           \item Orientar a progressão do portador da bola;
           \item Parar ou atrasar o ataque ou contra-ataque adversário;
           \item Propiciar maior tempo para organização defensiva;
           \item Restringir as possibilidades de passe a outro jogador adversário;
           \item Evitar o drible que favorece progressão pelo campo de jogo em direção ao gol;
           \item Impedir a finalização à baliza.
           \end{itemize}    \\
           \midrule
                \thead{Cobertura defensiva}
           \begin{itemize}
           \item Servir de novo obstáculo ao portador da bola, caso esse passe pelo jogador de contenção;
           \item Transmitir segurança e confiança ao jogador de contenção para que ele tenha iniciativa de combate às ações ofensivas do portador da bola;
           \end{itemize}    \\
           \midrule
                \thead{Equilíbrio}
           \begin{itemize}
           \item Assegurar a estabilidade defensiva da região de disputa da bola;
           \item Apoiar os companheiros que executam as ações de contenção e cobertura defensiva;
           \item Cobrir eventuais linhas de passe;
           \item Marcar potenciais jogadores que podem receber a bola;
           \item Fazer recuperação defensiva sobre o portador da bola;
           \item Recuperar ou afastar a bola da zona onde ela se encontra.
           \end{itemize}    \\
           \midrule
                \thead{Concentração}
           \begin{itemize}
           \item Aumentar a proteção ao gol;
           \item Condicionar o jogo ofensivo adversário para zonas de menor risco do campo de jogo;
           \item Propiciar aumento de pressão no centro do jogo.
           \end{itemize}    \\
           \midrule
                \thead{Unidade defensiva}
           \begin{itemize}
           \item Permitir a equipe defender em unidade ou em bloco;
           \item Garantir estabilidade espacial e sincronia dinâmica entre as linhas longitudinais e transversais da equipe em ações ofensivas;
           \item Diminuir a amplitude ofensiva da equipe adversária na sua largura e profundidade;
           \item Assegurar linhas orientadoras básicas que influenciam as atitudes e os comportamentos tático-técnicos dos jogadores que se posicionam fora do centro de jogo;
           \item Equilibrar ou reequilibrar constantemente a repartição de forças da organização defensiva consoante às situações momentâneas de jogo;
           \item Reduzir o espaço de jogo utilizando a regra do impedimento;
           \item Obstruir  possíveis linhas de passe para jogadores que se encontram  fora do centro de jogo;
           \item Possibilitar a participação em uma ação defensiva subsequente;
           \item Propiciar que mais jogadores se posicionem no centro de jogo.
           \end{itemize}
           \end{tabular}    \\
           \bottomrule
\end{tabularx}
\end{table}
\end{document}